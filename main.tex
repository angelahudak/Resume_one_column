%!TEX TS-program = xelatex

% use only XeLaTeX or LuaLaTeX to compile
\documentclass[a4paper]{comcv}

\usepackage[english]{babel}

\title{Angela Hudak Resume}
\fullname{Angela}{Hudak}{}
\cvtitle{Computer Engineer}
\smallskip
\phone{(732) 850-5097}{(732) 850-5097}
\email{angelahudak@mail.rit.edu}
\github{https://github.com/angelahudak}{angelahudak}
\linkedin{https://www.linkedin.com/in/angelahudak/}{angelahudak}
\currentdate{, 16 Feb 2023}

\begin{document}

%%%%%%%%%%%%%%%%%%%%%%%%%%%%%%% Education %%%%%%%%%%%%%%%%%%%%%%%%%%%%%%%

\section{Education}
\combosection{Rochester Institute of Technology}{B.S. in Computer Engineering}{August 2018 - May 2023}{\smallskip}

%%%%%%%%%%%%%%%%%%%%%%%%%%%%%%% Experience %%%%%%%%%%%%%%%%%%%%%%%%%%%%%%%

\section{Experience}
%%%%%%%%%%%%%
%Experience 1
%%%%%%%%%%%%%

\combosection{Poseidon Systems}{Product Development Engineer}{Jan 2023 - May 2023}{
\smallskip
    \begin{tightlist}
    \item Working on advancing the development and redesign of the AP2200 Industrial Data Logger.
    \item Done extensive research into new features and fixes for an optimal product design. 
    \item Designing the Printed Circuit Board in EAGLE, and testing in LTSPICE.
    \end{tightlist}
}


%%%%%%%%%%%%%
%Experience 2
%%%%%%%%%%%%%

\combosection{Amazon Web Services (AWS)}{Technical Writer}{May 2022 - August 2022}{
\smallskip
    \begin{tightlist}
    \item Converted the FreeRTOS.org Wordpress site to GitHub Flavored Markdown for GitHub Pages.
    \item Created a Python script utilizing the Pypandoc API to convert between file types.
    \item Wrote extensive documentation in HTML so the conversion process can be repeated with an easy user experience.
    \end{tightlist}
}


%%%%%%%%%%%%%
%Experience 3
%%%%%%%%%%%%%

\combosection{Bryx Inc.}{Full-Stack Embedded Systems Engineer}{September 2021 - May 2022}{
\smallskip
    \begin{tightlist}
    \item Designed in KiCAD, developed, and tested prototype Printed Circuit Boards.
    \item Validated circuit design and component compatibility using LTSPICE.
    \item Completed through hole (THT) soldering for board building and re-work.
    \item Utilized breadboards for prototypes, concept validation, and electrical debugging.
    \item Various projects using mechanical CAD and using / maintaining 3D printers.
    \item Wrote Arduino and Python code to interface and test hardware.
    \item Communicated with third party companies for hardware product development.
    \end{tightlist}
}


%%%%%%%%%%%%%
%Experience 4
%%%%%%%%%%%%%

\combosection{Abbott}{Project Support Engineer}{July 2020 - December 2020}{
\smallskip
    \begin{tightlist}
    \item Designed and developed the backend for a site that would allow the user to store and analyze data using Python and the Pandas library.
    \item The application was able to receive and organize data from a csv file so it could be easily viewed by the user.
    %\item Learned C++ to develop smaller applications which could potentially be used in upcoming projects.
    \end{tightlist}
}
\smallskip
% \vspace{\topsep}


%%%%%%%%%%%%%%%%%%%%%%%%%%%%%%% Projects %%%%%%%%%%%%%%%%%%%%%%%%%%%%%%%

\section{Projects}

%%%%%%%%%%%%%
%Project 1
%%%%%%%%%%%%%
\combosection{Gamification and Performance Monitoring of Sensorimotor Training}{}{August 2022 - Present}{}
\smallskip
    \begin{tightlist}
        \item A Multidisciplinary Senior Design project for NASA, in order to train and monitor Astronauts on their sense of balance.
        \item Performed extensive electrical documentation and validation for Tolomatic actuators, sensors, and kill switch circuits.
        \item Implemented an Arduino MEGA to read from four S-Type load cells inside of the move-able platform. Arduino C/C++ code calibrates, reads, and outputs data. A Python script reads the data and calculates a Balance score for the VR game and stores the raw coordinate calculations for MATLAB plotting / graphs. 
        \item Designed and developed a Printed Circuit Board (PCB) in KiCAD for attachable accelerometers.
        \item Implemented python server for communication between sensors, attachable accelerometers, actuators, and kill switch to the Unity VR game.

\end{tightlist}

% \combosection{MIPS Processor}{VHDL}{January 2021 - May 2021}{}
% \smallskip
%     \begin{tightlist}
%         \item This project was completed over the course of my Digital System Design 2 class.
%         \item Used Xilinx Vivado to create the functional parts of a MIPS Processor.
%         \item Constructed with the parts: ALU, Register File, Instruction Fetch, Decode, Execute, Memory, and Writeback Stages.
%         \item The processor was able to receive different MIPS assembly instructions and execute those commands.
% \end{tightlist}

%%%%%%%%%%%%%
%Project 2
%%%%%%%%%%%%%
% \vspace{\topsep}
% \combosection{Bomber Game}{6502 Assembly, Vim, Ubuntu}{April 2020}{\href{https://github.com/angelahudak/Bomber}{github.com/angelahudak/Bomber}}
% \vspace{\topsep}
%     \begin{tightlist}
%         \item A game developed for the Atari 2600 using the 6502 Assembly architecture.
%         \item Learned how to create sprites and digits on playing field using assembly commands.
%         \item The game contains a multitude of features such as a play field, scoreboard, missiles, two players on the screen at a time, warp border to the left and right, and background sound.
% \end{tightlist}

%%%%%%%%%%%%%
%Project 3
%%%%%%%%%%%%%

% \combosection{SPEX Rover Project}{Electronics Team Member}{January 2019 - April 2019}{\href{https://spex.rit.edu/projects/}{spex.rit.edu/projects}}
% \vspace{\topsep}
%     \begin{tightlist}
%         \item The Rover is a team project that is a part of the SPEX Club.
%         \item My responsibility was to connect the raspberry pi 3B+ to an xbox one controller through blue tooth and enable it so it stayed connected even when shut off.
%         \item Also taught others how to solder properly and safely.
%     \end{tightlist}
\smallskip
%%%%%%%%%%%%%%%%%%%%%%%%%%%%%%% Skills %%%%%%%%%%%%%%%%%%%%%%%%%%%%%%%

\section{Skills}
\begin{itemize}
    %Languages
    \item {\bf{Languages: }}  {Python, C, Arduino C/C++, C++, \LaTeX} 
    
    %Software
    \item {\bf{Software: }} {KiCAD, EAGLE, PSPICE / LTSPICE, VSCode, Raspbian, Vim, RedHat, Ubuntu}
    
    %Hardware
    \item {\bf{Hardware: }}  {Arduino, Raspberry pi, STM32L476xx, Breadboard / Circuitry, Oscilloscopes, Digital Multi-meter, Waveform Generator} 
\end{itemize}
\smallskip
%%%%%%%%%%%%%%%%%%%%%%%%%%%%%%% Organizations %%%%%%%%%%%%%%%%%%%%%%%%%%%%%%%

\section{Extracurriculars}

%%%%%%%%%%%%%
%Organizations 1
%%%%%%%%%%%%%
\combosection{Computer Science House}{House Improvements Director, 3D Admin}{August 2018 – January 2022}{\href{https://csh.rit.edu/}{csh.rit.edu}}
\vspace{\topsep}
\smallskip
\begin{tightlist}
    \item Computer Science House is a living and learning community with a helpful environment that emphasizes hands-on learning and projects outside of the classroom.

    \item The House Improvements director delegates projects that improve the physical aspects of floor, such as painting, cleaning, building, and organizing House's resources.

	\item A 3D Print Administrator assists and educates other members on how to print 3D files effectively and taking care of 3D printers.
	
\end{tightlist}

%%%%%%%%%%%%%
%Organizations 2
%%%%%%%%%%%%%

% \combosection{RIT Space Exploration Club (SPEX)}{Member}{January 2019 - December 2019}{\href{https://spex.rit.edu/}{spex.rit.edu}}
% \vspace{\topsep}
% \begin{tightlist}
%     \item Worked with other members on the rover electronics team by building a working prototype rover in preparation for the University Rover Challenge. 

% \end{tightlist}


\end{document}
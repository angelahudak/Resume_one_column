%!TEX TS-program = xelatex

% use only XeLaTeX or LuaLaTeX to compile
\documentclass[a4paper]{comcv}

\usepackage[english]{babel}

\title{Angela Hudak Resume}
\fullname{Angela}{Hudak}{}
\cvtitle{A Computer Engineering student seeking an internship for Summer 2022}
\smallskip
\phone{(732) 850-5097}{(732) 850-5097}
\email{angelahudak@mail.rit.edu}
\github{https://github.com/angelahudak}{angelahudak}
\linkedin{https://www.linkedin.com/in/angelahudak/}{angelahudak}
\currentdate{, 21 Feb 2022}

\begin{document}

%%%%%%%%%%%%%%%%%%%%%%%%%%%%%%% Education %%%%%%%%%%%%%%%%%%%%%%%%%%%%%%%

\section{Education}
\combosection{Rochester Institute of Technology}{B.S. in Computer Engineering}{August 2018 - May 2023}{\smallskip}

%%%%%%%%%%%%%%%%%%%%%%%%%%%%%%% Experience %%%%%%%%%%%%%%%%%%%%%%%%%%%%%%%

\section{Experience}
%%%%%%%%%%%%%
%Experience 1
%%%%%%%%%%%%%

\combosection{Bryx Inc.}{Full-Stack Embedded Engineering Intern}{September 2021 - May 2022}{
\smallskip
    \begin{tightlist}
    \item Hardware / PCB design
    \item Hole soldering and board re-work
    \item Electrical debugging and related lab work
    \item Buildup prototypes on a breadboard for quick concept validation
    \item Various projects using mechanical CAD and using / maintaining 3D printers
    \item Wrote basic Arduino code to interface with hardware
    \end{tightlist}
}

%%%%%%%%%%%%%
%Experience 2
%%%%%%%%%%%%%

\combosection{Abbott POC}{Project Support Engineering Intern}{July 2020 - December 2020}{
\smallskip
    \begin{tightlist}
    \item Worked in a remote setting with a Software Engineering Team.
    \item Designed and developed the backend that would allow the user to store and analyze data using Python and the Pandas library.
    \item The application was able to receive data from a csv file and would automate the organization of it so it could be easily viewed by the user.
    \item Learned C++ to develop smaller applications which could potentially be used in upcoming projects.
    \end{tightlist}
}

% \vspace{\topsep}

%%%%%%%%%%%%%%%%%%%%%%%%%%%%%%% Projects %%%%%%%%%%%%%%%%%%%%%%%%%%%%%%%

\section{Projects}

%%%%%%%%%%%%%
%Project 1
%%%%%%%%%%%%%

\combosection{MIPS Project}{VHDL}{January 2021 - May 2021}{}
\smallskip
    \begin{tightlist}
        \item This project was completed over the course of my Digital System Design 2 class.
        \item Used Xilinx Vivado to create the functional parts of a MIPS Processor.
        \item The parts that were included in the processor were ALU, Register File, Instruction Fetch, Decode, Execute, Memory, and Writeback Stages.
        \item The processor was able to receive different MIPS assembly instructions and execute those commands.
\end{tightlist}

%%%%%%%%%%%%%
%Project 2
%%%%%%%%%%%%%
\vspace{\topsep}
\combosection{Bomber Game}{6502 Assembly, Vim, Ubuntu}{April 2020}{\href{https://github.com/angelahudak/Bomber}{github.com/angelahudak/Bomber}}
\vspace{\topsep}
    \begin{tightlist}
        \item A game developed for the Atari 2600 using the 6502 Assembly architecture.
        \item Learned how to create sprites and digits on playing field using assembly commands.
        \item The game contains a multitude of features such as a play field, scoreboard, missiles, two players on the screen at a time, warp border to the left and right, and background sound.
\end{tightlist}

%%%%%%%%%%%%%
%Project 3
%%%%%%%%%%%%%

% \combosection{SPEX Rover Project}{Electronics Team Member}{January 2019 - April 2019}{\href{https://spex.rit.edu/projects/}{spex.rit.edu/projects}}
% \vspace{\topsep}
%     \begin{tightlist}
%         \item The Rover is a team project that is a part of the SPEX Club.
%         \item My responsibility was to connect the raspberry pi 3B+ to an xbox one controller through blue tooth and enable it so it stayed connected even when shut off.
%         \item Also taught others how to solder properly and safely.
%     \end{tightlist}

%%%%%%%%%%%%%%%%%%%%%%%%%%%%%%% Skills %%%%%%%%%%%%%%%%%%%%%%%%%%%%%%%

\section{Skills}
\begin{itemize}
    %Languages
    \item {\bf{Languages: }}  {Python, C, VHDL, Arduino C/C++, C++, ARM / MIPS Assembly, MATLAB, \LaTeX} 
    
    %Software
    \item {\bf{Software: }} {KiCAD, Keil $\mu$Vision, PSPICE / LTSPICE, Xilinx Vivado, Pycharm, Raspbian, Microsoft Office, Vim, RedHat, Ubuntu}
    
    %Hardware
    \item {\bf{Hardware: }}  {Arduino, Raspberry pi, MSP432p401r,  Basys3,  FRDM-KL46Z Board, Oscilloscopes, Breadboard / Circuitry, Digital Multi-meter, Waveform Generator} 
\end{itemize}

%%%%%%%%%%%%%%%%%%%%%%%%%%%%%%% Organizations %%%%%%%%%%%%%%%%%%%%%%%%%%%%%%%

\section{Extracurriculars}

%%%%%%%%%%%%%
%Organizations 1
%%%%%%%%%%%%%
\combosection{Computer Science House}{House Improvements Director, 3D Admin}{August 2018 – January 2022}{\href{https://csh.rit.edu/}{csh.rit.edu}}
\vspace{\topsep}
\begin{tightlist}
    \item Computer Science House is a living and learning community with a helpful environment that emphasizes hands-on learning and projects outside of the classroom.

    \item The House Improvements director delegates projects that improve the physical aspects of floor, such as painting, cleaning, building, and organizing House's resources.

	\item A 3D Print Administrator assists and educates other members on how to print 3D files effectively and taking care of 3D printers.
	
\end{tightlist}

%%%%%%%%%%%%%
%Organizations 2
%%%%%%%%%%%%%

\combosection{RIT Space Exploration Club (SPEX)}{Member}{January 2019 - December 2019}{\href{https://spex.rit.edu/}{spex.rit.edu}}
\vspace{\topsep}
\begin{tightlist}
    \item Worked with other members on the rover electronics team by building a working prototype rover in preparation for the University Rover Challenge. 

\end{tightlist}


\end{document}